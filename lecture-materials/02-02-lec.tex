% Options for packages loaded elsewhere
\PassOptionsToPackage{unicode}{hyperref}
\PassOptionsToPackage{hyphens}{url}
\PassOptionsToPackage{dvipsnames,svgnames,x11names}{xcolor}
%
\documentclass[
  11pt,
]{article}
\usepackage{amsmath,amssymb}
\usepackage{setspace}
\usepackage{iftex}
\ifPDFTeX
  \usepackage[T1]{fontenc}
  \usepackage[utf8]{inputenc}
  \usepackage{textcomp} % provide euro and other symbols
\else % if luatex or xetex
  \usepackage{unicode-math} % this also loads fontspec
  \defaultfontfeatures{Scale=MatchLowercase}
  \defaultfontfeatures[\rmfamily]{Ligatures=TeX,Scale=1}
\fi
\usepackage{lmodern}
\ifPDFTeX\else
  % xetex/luatex font selection
\fi
% Use upquote if available, for straight quotes in verbatim environments
\IfFileExists{upquote.sty}{\usepackage{upquote}}{}
\IfFileExists{microtype.sty}{% use microtype if available
  \usepackage[]{microtype}
  \UseMicrotypeSet[protrusion]{basicmath} % disable protrusion for tt fonts
}{}
\makeatletter
\@ifundefined{KOMAClassName}{% if non-KOMA class
  \IfFileExists{parskip.sty}{%
    \usepackage{parskip}
  }{% else
    \setlength{\parindent}{0pt}
    \setlength{\parskip}{6pt plus 2pt minus 1pt}}
}{% if KOMA class
  \KOMAoptions{parskip=half}}
\makeatother
\usepackage{xcolor}
\usepackage[margin=1in]{geometry}
\usepackage{color}
\usepackage{fancyvrb}
\newcommand{\VerbBar}{|}
\newcommand{\VERB}{\Verb[commandchars=\\\{\}]}
\DefineVerbatimEnvironment{Highlighting}{Verbatim}{commandchars=\\\{\}}
% Add ',fontsize=\small' for more characters per line
\usepackage{framed}
\definecolor{shadecolor}{RGB}{248,248,248}
\newenvironment{Shaded}{\begin{snugshade}}{\end{snugshade}}
\newcommand{\AlertTok}[1]{\textcolor[rgb]{0.94,0.16,0.16}{#1}}
\newcommand{\AnnotationTok}[1]{\textcolor[rgb]{0.56,0.35,0.01}{\textbf{\textit{#1}}}}
\newcommand{\AttributeTok}[1]{\textcolor[rgb]{0.13,0.29,0.53}{#1}}
\newcommand{\BaseNTok}[1]{\textcolor[rgb]{0.00,0.00,0.81}{#1}}
\newcommand{\BuiltInTok}[1]{#1}
\newcommand{\CharTok}[1]{\textcolor[rgb]{0.31,0.60,0.02}{#1}}
\newcommand{\CommentTok}[1]{\textcolor[rgb]{0.56,0.35,0.01}{\textit{#1}}}
\newcommand{\CommentVarTok}[1]{\textcolor[rgb]{0.56,0.35,0.01}{\textbf{\textit{#1}}}}
\newcommand{\ConstantTok}[1]{\textcolor[rgb]{0.56,0.35,0.01}{#1}}
\newcommand{\ControlFlowTok}[1]{\textcolor[rgb]{0.13,0.29,0.53}{\textbf{#1}}}
\newcommand{\DataTypeTok}[1]{\textcolor[rgb]{0.13,0.29,0.53}{#1}}
\newcommand{\DecValTok}[1]{\textcolor[rgb]{0.00,0.00,0.81}{#1}}
\newcommand{\DocumentationTok}[1]{\textcolor[rgb]{0.56,0.35,0.01}{\textbf{\textit{#1}}}}
\newcommand{\ErrorTok}[1]{\textcolor[rgb]{0.64,0.00,0.00}{\textbf{#1}}}
\newcommand{\ExtensionTok}[1]{#1}
\newcommand{\FloatTok}[1]{\textcolor[rgb]{0.00,0.00,0.81}{#1}}
\newcommand{\FunctionTok}[1]{\textcolor[rgb]{0.13,0.29,0.53}{\textbf{#1}}}
\newcommand{\ImportTok}[1]{#1}
\newcommand{\InformationTok}[1]{\textcolor[rgb]{0.56,0.35,0.01}{\textbf{\textit{#1}}}}
\newcommand{\KeywordTok}[1]{\textcolor[rgb]{0.13,0.29,0.53}{\textbf{#1}}}
\newcommand{\NormalTok}[1]{#1}
\newcommand{\OperatorTok}[1]{\textcolor[rgb]{0.81,0.36,0.00}{\textbf{#1}}}
\newcommand{\OtherTok}[1]{\textcolor[rgb]{0.56,0.35,0.01}{#1}}
\newcommand{\PreprocessorTok}[1]{\textcolor[rgb]{0.56,0.35,0.01}{\textit{#1}}}
\newcommand{\RegionMarkerTok}[1]{#1}
\newcommand{\SpecialCharTok}[1]{\textcolor[rgb]{0.81,0.36,0.00}{\textbf{#1}}}
\newcommand{\SpecialStringTok}[1]{\textcolor[rgb]{0.31,0.60,0.02}{#1}}
\newcommand{\StringTok}[1]{\textcolor[rgb]{0.31,0.60,0.02}{#1}}
\newcommand{\VariableTok}[1]{\textcolor[rgb]{0.00,0.00,0.00}{#1}}
\newcommand{\VerbatimStringTok}[1]{\textcolor[rgb]{0.31,0.60,0.02}{#1}}
\newcommand{\WarningTok}[1]{\textcolor[rgb]{0.56,0.35,0.01}{\textbf{\textit{#1}}}}
\usepackage{longtable,booktabs,array}
\usepackage{calc} % for calculating minipage widths
% Correct order of tables after \paragraph or \subparagraph
\usepackage{etoolbox}
\makeatletter
\patchcmd\longtable{\par}{\if@noskipsec\mbox{}\fi\par}{}{}
\makeatother
% Allow footnotes in longtable head/foot
\IfFileExists{footnotehyper.sty}{\usepackage{footnotehyper}}{\usepackage{footnote}}
\makesavenoteenv{longtable}
\usepackage{graphicx}
\makeatletter
\def\maxwidth{\ifdim\Gin@nat@width>\linewidth\linewidth\else\Gin@nat@width\fi}
\def\maxheight{\ifdim\Gin@nat@height>\textheight\textheight\else\Gin@nat@height\fi}
\makeatother
% Scale images if necessary, so that they will not overflow the page
% margins by default, and it is still possible to overwrite the defaults
% using explicit options in \includegraphics[width, height, ...]{}
\setkeys{Gin}{width=\maxwidth,height=\maxheight,keepaspectratio}
% Set default figure placement to htbp
\makeatletter
\def\fps@figure{htbp}
\makeatother
\setlength{\emergencystretch}{3em} % prevent overfull lines
\providecommand{\tightlist}{%
  \setlength{\itemsep}{0pt}\setlength{\parskip}{0pt}}
\setcounter{secnumdepth}{-\maxdimen} % remove section numbering
\usepackage{float}
\floatplacement{figure}{ht}
\usepackage[section]{placeins}
\usepackage{longtable}
\usepackage{hyperref}
\hypersetup{colorlinks = true, linkcolor = blue, urlcolor = blue}
\widowpenalty10000
\clubpenalty10000
\usepackage[page,header]{appendix}
\usepackage{titletoc}
\usepackage{tocloft}
\usepackage{makecell}
\ifLuaTeX
  \usepackage{selnolig}  % disable illegal ligatures
\fi
\IfFileExists{bookmark.sty}{\usepackage{bookmark}}{\usepackage{hyperref}}
\IfFileExists{xurl.sty}{\usepackage{xurl}}{} % add URL line breaks if available
\urlstyle{same}
\hypersetup{
  pdftitle={Lecture: Applied Bayesian Statistics II},
  colorlinks=true,
  linkcolor={blue},
  filecolor={Maroon},
  citecolor={Blue},
  urlcolor={Blue},
  pdfcreator={LaTeX via pandoc}}

\title{Lecture: Applied Bayesian Statistics II}
\author{}
\date{\vspace{-2.5em}}

\begin{document}
\maketitle

\setstretch{1.5}
\hypertarget{overview}{%
\subsection{Overview}\label{overview}}

\hypertarget{what-is-brms}{%
\subsubsection{What is brms?}\label{what-is-brms}}

The brms package provides an interface to fit Bayesian generalized (non-)linear multivariate multilevel models using Stan. The formula syntax is very similar to that of the package lme4 to provide a familiar and simple interface for performing regression analyses.

\href{https://paul-buerkner.github.io/brms}{paul-buerkner.github.io/brms}

It was created and is being maintained by \href{https://paul-buerkner.github.io/}{Paul Bürkner}. It is extensively documented on its own \href{https://paul-buerkner.github.io/brms/}{website}.

\hypertarget{comparison-pre-implemented-model-types}{%
\subsubsection{Comparison: Pre-implemented model types}\label{comparison-pre-implemented-model-types}}

\begin{center}\includegraphics[width=0.8\linewidth]{images/canned} \end{center}

Source: \href{https://cloud.r-project.org/web/packages/brms/vignettes/brms_overview.pdf}{Bürkner, Paul-Christian (2022). brms: An R Package for Bayesian Multilevel Models
using Stan.}

\emph{Note:} This table is from a 2022 publication. Newer versions of \texttt{brms} handle \href{https://cloud.r-project.org/web/packages/brms/vignettes/brms_missings.html}{missing values}.

\hypertarget{brms-vignettes-as-an-indication-of-its-versatility}{%
\subsubsection{brms vignettes as an indication of its versatility}\label{brms-vignettes-as-an-indication-of-its-versatility}}

\hypertarget{general}{%
\paragraph{General}\label{general}}

\begin{itemize}
\tightlist
\item
  \href{https://www.jstatsoft.org/article/view/v080i01}{General Introduction to brms}
\end{itemize}

\hypertarget{model-types}{%
\paragraph{Model types}\label{model-types}}

\begin{itemize}
\tightlist
\item
  \href{https://journal.r-project.org/archive/2018/RJ-2018-017/index.html}{Advanced Multilevel Modeling with brms}
\item
  \href{https://arxiv.org/abs/1905.09501}{Bayesian Item Response Modeling with brms}
\item
  \href{https://paul-buerkner.github.io/brms/articles/brms_distreg.html}{Estimating Distributional Models with brms}
\item
  \href{https://paul-buerkner.github.io/brms/articles/brms_multivariate.html}{Estimating Multivariate Models with brms}
\item
  \href{https://paul-buerkner.github.io/brms/articles/brms_nonlinear.html}{Estimating Non-Linear Models with brms}
\item
  \href{https://paul-buerkner.github.io/brms/articles/brms_phylogenetics.html}{Estimating Phylogenetic Multilevel Models with brms}
\end{itemize}

\hypertarget{auxiliary}{%
\paragraph{Auxiliary}\label{auxiliary}}

\begin{itemize}
\tightlist
\item
  \href{https://paul-buerkner.github.io/brms/articles/brms_customfamilies.html}{Define Custom Response Distributions with brms}
\item
  \href{https://paul-buerkner.github.io/brms/articles/brms_families.html}{Parameterization of Response Distributions in brms}
\item
  \href{https://paul-buerkner.github.io/brms/articles/brms_missings.html}{Handle Missing Values with brms}
\item
  \href{https://paul-buerkner.github.io/brms/articles/brms_monotonic.html}{Estimating Monotonic Effects with brms}
\item
  \href{https://paul-buerkner.github.io/brms/articles/brms_threading.html}{Running brms models with within-chain parallelization}
\end{itemize}

\hypertarget{a-function-call-to-brms}{%
\subsubsection{A function call to brms}\label{a-function-call-to-brms}}

\begin{Shaded}
\begin{Highlighting}[]
\NormalTok{lm\_brms }\OtherTok{\textless{}{-}}\NormalTok{ brms}\SpecialCharTok{::}\FunctionTok{brm}\NormalTok{(}
\NormalTok{  sup\_afd }\SpecialCharTok{\textasciitilde{}}                              \CommentTok{\# outcome}
\NormalTok{    la\_self,                             }\CommentTok{\# predictor}
  \AttributeTok{data =}\NormalTok{ gles,                           }\CommentTok{\# data}
  \AttributeTok{family =} \FunctionTok{gaussian}\NormalTok{(}\AttributeTok{link =} \StringTok{"identity"}\NormalTok{),  }\CommentTok{\# family and link}
  \AttributeTok{chains =}\NormalTok{ 4L,                           }\CommentTok{\# number of chains}
  \AttributeTok{iter =}\NormalTok{ 2000L,                          }\CommentTok{\# number of iterations per chain}
  \AttributeTok{warmup =}\NormalTok{ 1000L,                        }\CommentTok{\# number of warm{-}up samples per chain}
  \AttributeTok{algorithm =} \StringTok{"sampling"}\NormalTok{,                }\CommentTok{\# algorithm (HMC/NUTS)}
  \AttributeTok{backend =} \StringTok{"rstan"}\NormalTok{,                     }\CommentTok{\# backend (rstan)}
  \AttributeTok{seed =}\NormalTok{ 20231123L                       }\CommentTok{\# seed}
\NormalTok{)}
\end{Highlighting}
\end{Shaded}

\hypertarget{what-happens-under-the-hood}{%
\subsubsection{What happens under the hood}\label{what-happens-under-the-hood}}

\begin{center}\includegraphics[width=0.75\linewidth]{images/procedure} \end{center}

Source: \href{https://cloud.r-project.org/web/packages/brms/vignettes/brms_overview.pdf}{Bürkner, Paul-Christian (2022). brms: An R Package for Bayesian Multilevel Models
using Stan.}

\hypertarget{linear-model}{%
\subsection{Linear model}\label{linear-model}}

\hypertarget{likelihood}{%
\subsubsection{Likelihood}\label{likelihood}}

The linear model stipulates that the observed outcomes \(y_i\) for every unit \(i\) can be expressed as realizations from a normal distribution with unit-specific \emph{mean or location parameter} \(\mu_i\) and a constant (i.e., general) \emph{variance or scale parameter} \(\sigma^2\).

\[y_i \sim \text{N}(\mu_i, \sigma^2) \text{ for all }i = 1,...N\]

or, alternatively,

\[y_i = \mu_i + \epsilon_i  \text{ for all }i = 1,...N \\ \epsilon_i \sim \text{N}(0, \sigma^2)\]

The latter notation makes explicit that each observed \(y_i\) can be thought of as a combination of a \emph{systematic component}, \(\mu_i\), and a \emph{stochastic error component}, \(\epsilon_i\), which follows a zero-mean normal distribution with constant variance \(\sigma^2\).

\hypertarget{the-systematic-component}{%
\subsubsection{The systematic component}\label{the-systematic-component}}

The systematic component is represented by the mean parameter \(\mu_i\). In fact, \(\mu_i\) is merely a \emph{transformed parameter}: It is a linear function of unit-specific data \(\mathbf{x}_i\) and coefficients \(\beta\).

The formula below illustrates this, using the row vector notation \(\mathbf{x}_i^{\prime} \beta\) as shorthand for the scalar notation \(\beta_1 + \beta_2 x_{i, 2} + ...+\beta_k x_{i,k}\).

\[\mu_i = \underbrace{\mathbf{x}_i^{\prime} \beta}_{= \beta_1 + \beta_2 x_{i, 2} + ...+\beta_k x_{i,k}}  \text{ for all }i = 1,...N\]

\hypertarget{parameters-and-priors}{%
\subsubsection{Parameters and priors}\label{parameters-and-priors}}

In the linear model, all coefficients \(\beta\) as well as the variance \(\sigma^2\) are model parameters.

In Bayesian analysis, we must assign them priors (though \texttt{brms}, like Stan, will assign default uniform priors if we do not explicitly specify priors).

\hypertarget{data}{%
\subsubsection{Data}\label{data}}

We model respondents' support for the AfD (\texttt{sup\_afd}, measured on an 11-point scale ranging from -5 to 5) as a function of respondents' pro-redistribution preferences (\texttt{se\_self}) and anti-immigration preferences (\texttt{la\_self}), a multiplicative interaction term between the two, and some controls: Gender (\texttt{fem}), age (\texttt{age}), and East/West residence (\texttt{east}).

Both \texttt{se\_self} and \texttt{la\_self} are measured on 11-point scales:

\begin{itemize}
\tightlist
\item
  \texttt{se\_self} ranges from values (0) ``less taxes and deductions, even if that means less social spending'' to (10) ``more social spending, even if that means more taxes and deductions''.
\item
  \texttt{la\_self} ranges from values (0) ``facilitate immigration'' to (10) ``restrict immigration''.
\end{itemize}

The model formula is given by

\[
\mathtt{sup\_afd}_i = \\\beta_1 + \beta_2 \mathtt{se\_self}_i + \beta_3 \mathtt{la\_self}_i + \\ \beta_4 \mathtt{fem}_i + \beta_5 \mathtt{east}_i + \beta_6 \mathtt{age}_i + \\ \beta_7 \mathtt{se\_self}_i \times \mathtt{la\_self}_i + \epsilon
\]

\hypertarget{fitting}{%
\subsection{Fitting}\label{fitting}}

\hypertarget{choosing-priors}{%
\subsubsection{Choosing priors}\label{choosing-priors}}

\texttt{brms} uses default priors for certain ``classes'' of parameters. To check these defaults,
we need to supply the model formula, data, and generative model (i.e., family and link function) to
\texttt{brms::get\_prior()}.

\begin{Shaded}
\begin{Highlighting}[]
\CommentTok{\# Get default priors}
\NormalTok{default\_priors }\OtherTok{\textless{}{-}}\NormalTok{ brms}\SpecialCharTok{::}\FunctionTok{get\_prior}\NormalTok{(}
\NormalTok{  sup\_afd }\SpecialCharTok{\textasciitilde{}}                              \CommentTok{\# outcome}
\NormalTok{    la\_self }\SpecialCharTok{*}                            \CommentTok{\# immigration preferences}
\NormalTok{    se\_self }\SpecialCharTok{+}                            \CommentTok{\# redistribution preferences}
\NormalTok{    fem }\SpecialCharTok{+}                                \CommentTok{\# gender}
\NormalTok{    east }\SpecialCharTok{+}                               \CommentTok{\# east/west residence}
\NormalTok{    age,                                 }\CommentTok{\# age}
  \AttributeTok{data =}\NormalTok{ gles,                           }\CommentTok{\# data}
  \AttributeTok{family =} \FunctionTok{gaussian}\NormalTok{(}\AttributeTok{link =} \StringTok{"identity"}\NormalTok{)   }\CommentTok{\# family and link}
\NormalTok{)}
\NormalTok{default\_priors}
\end{Highlighting}
\end{Shaded}

\emph{Note:} Missing entries in the \texttt{prior} column denote flat/uniform priors.

\hypertarget{define-custom-priors}{%
\subsubsection{Define custom priors}\label{define-custom-priors}}

If we don't like the default priors, we can create a \texttt{brmsprior} object
by specifying the desired distributional properties of parameters of
various classes:

\begin{Shaded}
\begin{Highlighting}[]
\NormalTok{custom\_priors }\OtherTok{\textless{}{-}} \FunctionTok{c}\NormalTok{(}
\NormalTok{  brms}\SpecialCharTok{::}\FunctionTok{prior}\NormalTok{(}\FunctionTok{normal}\NormalTok{(}\DecValTok{0}\NormalTok{, }\DecValTok{5}\NormalTok{), }\AttributeTok{class =}\NormalTok{ b),          }\CommentTok{\# normal slopes}
\NormalTok{  brms}\SpecialCharTok{::}\FunctionTok{prior}\NormalTok{(}\FunctionTok{normal}\NormalTok{(}\DecValTok{0}\NormalTok{, }\DecValTok{5}\NormalTok{), }\AttributeTok{class =}\NormalTok{ Intercept),  }\CommentTok{\# normal intercept}
\NormalTok{  brms}\SpecialCharTok{::}\FunctionTok{prior}\NormalTok{(}\FunctionTok{cauchy}\NormalTok{(}\DecValTok{0}\NormalTok{, }\DecValTok{5}\NormalTok{), }\AttributeTok{class =}\NormalTok{ sigma)       }\CommentTok{\# half{-}cauchy SD}
\NormalTok{)}
\NormalTok{custom\_priors}
\end{Highlighting}
\end{Shaded}

Let's think about these values intuitively. How informative/vague are our priors?

\hypertarget{prior-predictive-checks-fitting}{%
\subsubsection{Prior predictive checks: Fitting}\label{prior-predictive-checks-fitting}}

\begin{Shaded}
\begin{Highlighting}[]
\NormalTok{lm\_brms\_prior\_only }\OtherTok{\textless{}{-}}\NormalTok{ brms}\SpecialCharTok{::}\FunctionTok{brm}\NormalTok{(}
\NormalTok{  sup\_afd }\SpecialCharTok{\textasciitilde{}}                              \CommentTok{\# outcome}
\NormalTok{    la\_self }\SpecialCharTok{*}                            \CommentTok{\# immigration preferences}
\NormalTok{    se\_self }\SpecialCharTok{+}                            \CommentTok{\# redistribution preferences}
\NormalTok{    fem }\SpecialCharTok{+}                                \CommentTok{\# gender}
\NormalTok{    east }\SpecialCharTok{+}                               \CommentTok{\# east/west residence}
\NormalTok{    age,                                 }\CommentTok{\# age}
  \AttributeTok{data =}\NormalTok{ gles,                           }\CommentTok{\# data}
  \AttributeTok{family =} \FunctionTok{gaussian}\NormalTok{(}\AttributeTok{link =} \StringTok{"identity"}\NormalTok{),  }\CommentTok{\# family and link}
  \AttributeTok{prior =}\NormalTok{ custom\_priors,                 }\CommentTok{\# priors}
  \AttributeTok{sample\_prior =} \StringTok{"only"}\NormalTok{,                 }\CommentTok{\# samply only from prior}
  \AttributeTok{chains =}\NormalTok{ 2L,                           }\CommentTok{\# number of chains}
  \AttributeTok{iter =}\NormalTok{ 1000L,                          }\CommentTok{\# number of iterations per chain}
  \AttributeTok{warmup =}\NormalTok{ 0L,                           }\CommentTok{\# number of warm{-}up samples per chain}
  \AttributeTok{algorithm =} \StringTok{"sampling"}\NormalTok{,                }\CommentTok{\# algorithm (HMC/NUTS)}
  \AttributeTok{backend =} \StringTok{"rstan"}\NormalTok{,                     }\CommentTok{\# backend (rstan)}
  \AttributeTok{seed =}\NormalTok{ 20231123L                       }\CommentTok{\# seed}
\NormalTok{)}
\end{Highlighting}
\end{Shaded}

\hypertarget{check-model-predictions-vis-uxe0-vis-empirical-distribution-of-outcome}{%
\subsubsection{Check model predictions vis-à-vis empirical distribution of outcome}\label{check-model-predictions-vis-uxe0-vis-empirical-distribution-of-outcome}}

\begin{Shaded}
\begin{Highlighting}[]
\NormalTok{brms}\SpecialCharTok{::}\FunctionTok{pp\_check}\NormalTok{(lm\_brms\_prior\_only, }\AttributeTok{ndraws =} \DecValTok{100}\NormalTok{, }\AttributeTok{type =} \StringTok{"dens\_overlay"}\NormalTok{) }\SpecialCharTok{+}
  \FunctionTok{xlim}\NormalTok{(}\SpecialCharTok{{-}}\DecValTok{150}\NormalTok{, }\DecValTok{150}\NormalTok{)}
\end{Highlighting}
\end{Shaded}

\begin{verbatim}
## Warning: Removed 33045 rows containing non-finite values (`stat_density()`).
\end{verbatim}

\begin{center}\includegraphics{02-02-lec_files/figure-latex/brms-priorpc-2-1} \end{center}

\hypertarget{fitting-the-model}{%
\subsubsection{Fitting the model}\label{fitting-the-model}}

Lastly, we can fit the model using \texttt{brms::brm()}.

\emph{Note:} Model compilation and estimation may take a while.

\begin{Shaded}
\begin{Highlighting}[]
\NormalTok{lm\_brms }\OtherTok{\textless{}{-}}\NormalTok{ brms}\SpecialCharTok{::}\FunctionTok{brm}\NormalTok{(}
\NormalTok{  sup\_afd }\SpecialCharTok{\textasciitilde{}}                              \CommentTok{\# outcome}
\NormalTok{    la\_self }\SpecialCharTok{*}                            \CommentTok{\# immigration preferences}
\NormalTok{    se\_self }\SpecialCharTok{+}                            \CommentTok{\# redistribution preferences}
\NormalTok{    fem }\SpecialCharTok{+}                                \CommentTok{\# gender}
\NormalTok{    east }\SpecialCharTok{+}                               \CommentTok{\# east/west residence}
\NormalTok{    age,                                 }\CommentTok{\# age}
  \AttributeTok{data =}\NormalTok{ gles,                           }\CommentTok{\# data}
  \AttributeTok{family =} \FunctionTok{gaussian}\NormalTok{(}\AttributeTok{link =} \StringTok{"identity"}\NormalTok{),  }\CommentTok{\# family and link}
  \AttributeTok{prior =}\NormalTok{ custom\_priors,                 }\CommentTok{\# priors}
  \AttributeTok{chains =}\NormalTok{ 4L,                           }\CommentTok{\# number of chains}
  \AttributeTok{iter =}\NormalTok{ 2000L,                          }\CommentTok{\# number of iterations per chain}
  \AttributeTok{warmup =}\NormalTok{ 1000L,                        }\CommentTok{\# number of warm{-}up samples per chain}
  \AttributeTok{algorithm =} \StringTok{"sampling"}\NormalTok{,                }\CommentTok{\# algorithm (HMC/NUTS)}
  \AttributeTok{backend =} \StringTok{"rstan"}\NormalTok{,                     }\CommentTok{\# backend (rstan)}
  \AttributeTok{seed =}\NormalTok{ 20231123L                       }\CommentTok{\# seed}
\NormalTok{)}
\end{Highlighting}
\end{Shaded}

\hypertarget{summarize-and-diagnose}{%
\subsection{Summarize and diagnose}\label{summarize-and-diagnose}}

\hypertarget{model-summary-and-generic-diagnostics}{%
\subsubsection{Model summary and generic diagnostics}\label{model-summary-and-generic-diagnostics}}

First, we print the model summary. We can check \texttt{Rhat} for any signs of non-convergence.

\begin{verbatim}
##  Family: gaussian 
##   Links: mu = identity; sigma = identity 
## Formula: sup_afd ~ la_self * se_self + fem + east + age 
##    Data: gles (Number of observations: 1321) 
##   Draws: 4 chains, each with iter = 2000; warmup = 1000; thin = 1;
##          total post-warmup draws = 4000
## 
## Population-Level Effects: 
##                 Estimate Est.Error l-95% CI u-95% CI Rhat Bulk_ESS Tail_ESS
## Intercept          -3.81      0.43    -4.63    -2.95 1.00     2719     2473
## la_self             0.30      0.05     0.20     0.41 1.00     2474     2331
## se_self            -0.12      0.06    -0.25     0.01 1.00     2575     2151
## fem1               -0.59      0.14    -0.87    -0.32 1.00     4804     2719
## east1               0.42      0.15     0.14     0.71 1.00     4394     2740
## age                -0.01      0.00    -0.02    -0.01 1.00     5887     3038
## la_self:se_self     0.01      0.01    -0.01     0.03 1.00     2483     2183
## 
## Family Specific Parameters: 
##       Estimate Est.Error l-95% CI u-95% CI Rhat Bulk_ESS Tail_ESS
## sigma     2.44      0.05     2.36     2.53 1.00     5010     3136
## 
## Draws were sampled using sampling(NUTS). For each parameter, Bulk_ESS
## and Tail_ESS are effective sample size measures, and Rhat is the potential
## scale reduction factor on split chains (at convergence, Rhat = 1).
\end{verbatim}

\hypertarget{visual-diagnostics}{%
\subsubsection{Visual diagnostics}\label{visual-diagnostics}}

Let's explore the following visualizations of common generic diagnostics:

\begin{Shaded}
\begin{Highlighting}[]
\NormalTok{brms}\SpecialCharTok{::}\FunctionTok{mcmc\_plot}\NormalTok{(lm\_brms, }\AttributeTok{type =} \StringTok{"rhat"}\NormalTok{)  }\CommentTok{\# Gelman{-}Rubin }
\end{Highlighting}
\end{Shaded}

\includegraphics{02-02-lec_files/figure-latex/brms-visual-1.pdf}

\begin{Shaded}
\begin{Highlighting}[]
\NormalTok{brms}\SpecialCharTok{::}\FunctionTok{mcmc\_plot}\NormalTok{(lm\_brms, }\AttributeTok{type =} \StringTok{"acf"}\NormalTok{)   }\CommentTok{\# Autocorrelation}
\end{Highlighting}
\end{Shaded}

\begin{verbatim}
## Warning: The `facets` argument of `facet_grid()` is deprecated as of ggplot2 2.2.0.
## i Please use the `rows` argument instead.
## i The deprecated feature was likely used in the bayesplot package.
##   Please report the issue at <https://github.com/stan-dev/bayesplot/issues/>.
## This warning is displayed once every 8 hours.
## Call `lifecycle::last_lifecycle_warnings()` to see where this warning was
## generated.
\end{verbatim}

\includegraphics{02-02-lec_files/figure-latex/brms-visual-2.pdf}

\begin{Shaded}
\begin{Highlighting}[]
\NormalTok{brms}\SpecialCharTok{::}\FunctionTok{mcmc\_plot}\NormalTok{(lm\_brms, }\AttributeTok{type =} \StringTok{"trace"}\NormalTok{) }\CommentTok{\# Trace plots}
\end{Highlighting}
\end{Shaded}

\begin{verbatim}
## No divergences to plot.
\end{verbatim}

\includegraphics{02-02-lec_files/figure-latex/brms-visual-3.pdf}

See \texttt{help(mcmc\_plot)} for additional types of plots.

\hypertarget{algorithm-specific-diagnostics}{%
\subsubsection{Algorithm-specific diagnostics}\label{algorithm-specific-diagnostics}}

Note: We rely on the \texttt{check\_hmc\_diagnostics()} function from the \texttt{rstan} package. To ensure it works, we must extract the \texttt{stanfit} object nested in our \texttt{brmsfit} object via \texttt{lm\_brms\$fit}.

\begin{Shaded}
\begin{Highlighting}[]
\NormalTok{rstan}\SpecialCharTok{::}\FunctionTok{check\_hmc\_diagnostics}\NormalTok{(lm\_brms}\SpecialCharTok{$}\NormalTok{fit)}
\end{Highlighting}
\end{Shaded}

\begin{verbatim}
## 
## Divergences:
\end{verbatim}

\begin{verbatim}
## 0 of 4000 iterations ended with a divergence.
\end{verbatim}

\begin{verbatim}
## 
## Tree depth:
\end{verbatim}

\begin{verbatim}
## 0 of 4000 iterations saturated the maximum tree depth of 10.
\end{verbatim}

\begin{verbatim}
## 
## Energy:
\end{verbatim}

\begin{verbatim}
## E-BFMI indicated no pathological behavior.
\end{verbatim}

\hypertarget{what-if-i-find-signs-of-non-convergence}{%
\subsubsection{What if I find signs of non-convergence?}\label{what-if-i-find-signs-of-non-convergence}}

The standard answer is: Increase the length of your chains. It may especially help with warnings about Rhat, ESS, and low BFMI.

But this is not always the optimal strategy, and it may not solve your problem.

So, suppose that after running longer chains, one or several of the following still apply:

\begin{itemize}
\tightlist
\item
  Your algorithm-specific diagnostics throw warnings (that won't go away)
\item
  Your convergence diagnostics indicate signs of non-convergence (and increasing the warm-up period doesn't help)
\item
  Your algorithm is painfully slow
\end{itemize}

\hypertarget{dealing-with-non-covergence-and-computational-problems}{%
\subsubsection{Dealing with non-covergence and computational problems}\label{dealing-with-non-covergence-and-computational-problems}}

Here are some answers, partly based on \href{https://arxiv.org/abs/2011.01808}{Gelman et al.~(2020)} and the Stan Development Team's guide \href{https://mc-stan.org/misc/warnings.html}{\emph{Runtime warnings and convergence problems}}:

\begin{itemize}
\tightlist
\item
  Do read \href{https://mc-stan.org/misc/warnings.html}{\emph{Runtime warnings and convergence problems}}. It can help you understand a specific problem and potential solutions.
\item
  Do you get algorithm-specific warnings about divergences/max\_treedepth? Adjust the HMC/NUTS control arguments \texttt{adapt\_delta}, \texttt{stepsize}, and/or \texttt{max\_treedepth}.
\item
  Check if your model is well specified (e.g., do you have problems of separation in logistic regression?)
\item
  Adopt an efficient workflow for debugging:

  \begin{itemize}
  \tightlist
  \item
    Reduce model complexity. \href{https://hyunjimoon.github.io/SBC/articles/small_model_workflow.html}{Start with a simpler specification, gradually build up.} See where things start to go wrong.
  \item
    Use smaller sets of data, few chains, and short runs.
  \end{itemize}
\item
  Optimize priors:

  \begin{itemize}
  \tightlist
  \item
    If you use custom priors: Do your priors allow for posterior density in regions where you'd expect it?
  \item
    If you use default flat or very vague priors: Use stronger priors (within reason)
  \end{itemize}
\end{itemize}

\hypertarget{interpretation-quantities-of-interest}{%
\subsection{Interpretation: Quantities of interest}\label{interpretation-quantities-of-interest}}

\hypertarget{brms-functions}{%
\subsubsection{brms functions}\label{brms-functions}}

\texttt{brms} offers pre-implemented functions for plotting conditional expectations (aka expected values, linear predictions, or adjusted predictions).

Below, for instance, are the expected values of AfD support for men (0) and women (1).

\begin{Shaded}
\begin{Highlighting}[]
\NormalTok{brms}\SpecialCharTok{::}\FunctionTok{conditional\_effects}\NormalTok{(lm\_brms,}
                          \AttributeTok{effects =} \FunctionTok{c}\NormalTok{(}\StringTok{"fem"}\NormalTok{),}
                          \AttributeTok{points =} \ConstantTok{TRUE}\NormalTok{)}
\end{Highlighting}
\end{Shaded}

\includegraphics{02-02-lec_files/figure-latex/brms-marginal-effects-1.pdf}

\hypertarget{limits-of-brms-functions}{%
\subsubsection{Limits of brms functions}\label{limits-of-brms-functions}}

However, brms' functions are somewhat limited for more complex quantities of interest.

For instance, the continuous-by-continuous interaction
of \texttt{la\_self} and \texttt{se\_self}, \texttt{brms::conditional\_effects()} will give you the conditional expectation of \texttt{sup\_afd} as a function of \texttt{la\_self} at three characteristic values of \texttt{se\_self} (\texttt{mean(se\_self)\ +\ c(-1,\ 0,\ 1)\ *\ sd(se\_self)}), fixing all else at mean values:

\begin{Shaded}
\begin{Highlighting}[]
\NormalTok{brms}\SpecialCharTok{::}\FunctionTok{conditional\_effects}\NormalTok{(lm\_brms,}
                          \AttributeTok{effects =} \FunctionTok{c}\NormalTok{(}\StringTok{"la\_self:se\_self"}\NormalTok{))}
\end{Highlighting}
\end{Shaded}

\includegraphics{02-02-lec_files/figure-latex/brms-conditional-effects-1.pdf}

What we would really like to get, however, is the conditional marginal effect of \texttt{la\_self}, which shows how the effect of \texttt{la\_self} changes as a function of the values of the moderator \texttt{se\_self} (see \href{https://www.jstor.org/stable/25791835}{Brambor et. al, 2006}).

\hypertarget{marginaleffects}{%
\subsubsection{marginaleffects}\label{marginaleffects}}

The \href{https://marginaleffects.com/}{\texttt{marginaleffects}} package (``Predictions, Comparisons, Slopes, Marginal Means, and Hypothesis Tests''), developed by \href{https://arelbundock.com/}{Vincent Arel-Bundock}, allows users to \emph{``compute and plot predictions, slopes, marginal means, and comparisons (contrasts, risk ratios, odds, etc.) for over 100 classes of statistical and machine learning models in R''}.

A few years ago, compatibility with \texttt{brms} models was added.

\hypertarget{conditional-marginal-effects-plot}{%
\subsubsection{Conditional marginal effects plot}\label{conditional-marginal-effects-plot}}

\begin{Shaded}
\begin{Highlighting}[]
\NormalTok{marginaleffects}\SpecialCharTok{::}\FunctionTok{plot\_slopes}\NormalTok{(lm\_brms,}
                             \AttributeTok{variable =} \StringTok{"la\_self"}\NormalTok{,}
                             \AttributeTok{condition =} \StringTok{"se\_self"}\NormalTok{) }\SpecialCharTok{+}
\NormalTok{  ggplot2}\SpecialCharTok{::}\FunctionTok{xlab}\NormalTok{(}\FunctionTok{paste}\NormalTok{(}\StringTok{"Socio{-}economic policy preferences"}\NormalTok{,}
                      \StringTok{"(right{-}to{-}left)"}\NormalTok{,}
                      \AttributeTok{sep =} \StringTok{"}\SpecialCharTok{\textbackslash{}n}\StringTok{"}\NormalTok{)) }\SpecialCharTok{+}
\NormalTok{  ggplot2}\SpecialCharTok{::}\FunctionTok{ylab}\NormalTok{(}
    \FunctionTok{paste}\NormalTok{(}
      \StringTok{"Marginal effect of right{-}wing immigration"}\NormalTok{,}
      \StringTok{"policy preferences on AfD support"}\NormalTok{,}
      \AttributeTok{sep =} \StringTok{"}\SpecialCharTok{\textbackslash{}n}\StringTok{"}
\NormalTok{    )}
\NormalTok{  )}
\end{Highlighting}
\end{Shaded}

\includegraphics{02-02-lec_files/figure-latex/mfx-la-1.pdf}

\hypertarget{further-reading}{%
\subsubsection{Further reading}\label{further-reading}}

See the \texttt{marginaleffects} \href{https://marginaleffects.com/articles/brms.html}{Case Study 8: Bayes} for a complete overview of the package's compatibility with \texttt{brms}.

\hypertarget{posterior-predictive-checks}{%
\subsection{Posterior predictive checks}\label{posterior-predictive-checks}}

Posterior predictive checks involve simulating the data-generating process to
obtain replicated data given the estimated model. They can help us determine
how well our model fits the data.

This usually involves two questions:

\begin{enumerate}
\def\labelenumi{\arabic{enumi}.}
\tightlist
\item
  Does the \emph{family} yield an adequate generative model?

  \begin{itemize}
  \tightlist
  \item
    Does a Gaussian (normal) data-generating processes produce realistic replications of the observed values of \texttt{sup\_afd} (support for the AfD on the -5 to +5 scale)?
  \item
    Does the simulated distribution of the replications match the observed distribution of the outcome in the \emph{sample}?
  \end{itemize}
\item
  Does the \emph{systematic component} accurately predict outcomes?

  \begin{itemize}
  \tightlist
  \item
    Do our predictors accurately predict which individuals are more likely to support the AfD?
  \item
    How large is the \emph{observation-level discrepancy} between simulated replications and observed data?
  \end{itemize}
\end{enumerate}

\hypertarget{distributional-congruence}{%
\subsubsection{Distributional congruence}\label{distributional-congruence}}

To check whether the generative model produces distributions of replicated outcomes that match the distribution of the observed outcome, we can compare the density of the observed outcome with those of, say, \texttt{ndraws\ =\ 100} simulations. Each simulation is based on one post-warm-up sample from the posterior distribution.

\begin{Shaded}
\begin{Highlighting}[]
\NormalTok{brms}\SpecialCharTok{::}\FunctionTok{pp\_check}\NormalTok{(lm\_brms, }\AttributeTok{ndraws =} \DecValTok{100}\NormalTok{, }\AttributeTok{type =} \StringTok{"dens\_overlay"}\NormalTok{)}
\end{Highlighting}
\end{Shaded}

\begin{center}\includegraphics{02-02-lec_files/figure-latex/brms-lm-pp-1-1} \end{center}

So, what do you think?

\hypertarget{observation-level-prediction-error}{%
\subsubsection{Observation-level prediction error}\label{observation-level-prediction-error}}

To check the predictive accuracy of the model, we can investigate the distribution of observation-level prediction errors. A model with perfect fit would produce an error of \(0\) for all \(N\) observations.

Below, you see the distribution of errors for our linear model. What do you think?

\begin{Shaded}
\begin{Highlighting}[]
\NormalTok{brms}\SpecialCharTok{::}\FunctionTok{pp\_check}\NormalTok{(lm\_brms, }\AttributeTok{ndraws =} \DecValTok{1}\NormalTok{, }\AttributeTok{type =} \StringTok{"error\_hist"}\NormalTok{)}
\end{Highlighting}
\end{Shaded}

\begin{verbatim}
## `stat_bin()` using `bins = 30`. Pick better value with `binwidth`.
\end{verbatim}

\begin{center}\includegraphics{02-02-lec_files/figure-latex/brms-lm-pp-2-1} \end{center}

\hypertarget{comparison-the-model-as-a-zero-one-inflated-beta-zoib-regression}{%
\subsubsection{Comparison the model as a zero-one-inflated beta (ZOIB) regression}\label{comparison-the-model-as-a-zero-one-inflated-beta-zoib-regression}}

Zero-one-inflated beta (ZOIB) regression models bounded continuous outcomes on the
unit (i.e., \([0,1]\)) interval.
The ZOIB model is a GLM with a multi-family likelihood, meaning
that its likelihood is composed of a mixture of several constitutive likelihoods.
Specifically, it supplements a beta pdf for values \(y \in ]0, 1[\) with
additional pmfs for the boundary values \(y \in \{0,1\}\).

To model a bounded continuous outcome on the unit interval, we must transform the scale of AfD support to range
from 0 to 1 (with midpoint 0.5) instead of -5 to +5 (with midpoint 0), but we will scale it back later on.

We first observe the distributional congruence of the ZOIB-generated outcome
simulations.

\begin{Shaded}
\begin{Highlighting}[]
\NormalTok{brms}\SpecialCharTok{::}\FunctionTok{pp\_check}\NormalTok{(zoib\_brms, }\AttributeTok{ndraws =} \DecValTok{100}\NormalTok{, }\AttributeTok{type =} \StringTok{"dens\_overlay"}\NormalTok{)}
\end{Highlighting}
\end{Shaded}

\begin{center}\includegraphics{02-02-lec_files/figure-latex/brms-zoib-pp-1-1} \end{center}

We then turn to checking observation-level prediction errors.

\begin{Shaded}
\begin{Highlighting}[]
\NormalTok{brms}\SpecialCharTok{::}\FunctionTok{pp\_check}\NormalTok{(zoib\_brms, }\AttributeTok{ndraws =} \DecValTok{1}\NormalTok{, }\AttributeTok{type =} \StringTok{"error\_hist"}\NormalTok{)}
\end{Highlighting}
\end{Shaded}

\begin{verbatim}
## `stat_bin()` using `bins = 30`. Pick better value with `binwidth`.
\end{verbatim}

\begin{center}\includegraphics{02-02-lec_files/figure-latex/brms-zoib-pp-2-1} \end{center}

What do you conclude? Does the ZOIB-family accurately model the observed
sample-level distribution of the outcome? Are you happy with the predictive
accuracy of our current systematic component?

\end{document}
